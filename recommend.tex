\chapter{推薦システム}
\label{chap:recommend}
本章ではRive Serverに内包されている候補単語
推薦システムについて述べる。

\newpage
\section{概要}
ユーザーのコンテキストと入力を受け取り、
過去の全ての人の入力情報と比べることによって
ユーザーの入力する単語を予測し推薦するシステム。
クライアントを起動した瞬間から、
一つの操作(キーボードにタッチすることや候補単語へのタッチすること)
ごとに取得できる全てのコンテキストを入力と紐付けて
サーバーへ送信して、推薦システムがそのコンテキストにあった
推薦候補単語をクライアントへ送信する。
このプロセスを繰り返すことで本推薦システムは成り立っている。
本論文においては推薦候補単語とは
本推薦システムを通してでた結果から取得する
候補単語のことであり、
一般の仮名漢字変換や予測入力における候補単語は含まない。

\section{取得コンテキスト}
\label{sec:getcontext}
ここでは推薦システムにおいて使用しているコンテキストについて述べる。
コンテキストを何度も取得することで本システは成り立っているが、
入力が開始されてから終了まで変わらないコンテキスト何度も取得すると
システムに大きな負荷がかかるため、
入力開始から終了までコンテキストが
変化するものと変化しないないものに分類し実装することで負荷を軽減に役立てた。
入力が開始してから変化しないコンテキストを静的コンテキスト。
入力が開始してから変化する可能性があるコンテキストを動的コンテキストとした。
コンテキストにはユーザーが設定画面において設定するものと、
Rive日本語入力が自動的に取得するものがある。

\subsection{静的コンテキスト}
\label{staticcontext}
以下静的コンテキストとして利用しているものを述べる。

\subsubsection{アクティビティー}
\label{activity}
アクティビティーとは入力する対象のアプリケーションことである。
AndroidOSにおいてはメールクライアントであったり、
WEBブラウザアプリなどである。
アクテビティーをコンテキストに含めることにより、
サービスごとに入力されやすい単語を推薦しやすくする。

\subsubsection{性別}
性別をコンテキストに含めることにより、
同じ性別の人が使っている単語を推薦されやすくなる。
要素はユーザーが以下から選択する。
\begin{itemize}
  \item 男性
  \item 女性
  \item その他
\end{itemize}

\subsubsection{年齢}
年齢をコンテキストに含めることにより、
同じ年齢あるいは近い年齢の人が使っている単語を推薦しやすくする。

\subsubsection{キャラクター}
キャラクターをコンテキストに含めることにより、
簡易なユーザークラスタリングをし、
同じクラスタの人が使っている
単語を推薦しやすくする。
クラスタとは同様の嗜好を持った人々をまとめたグループのことである。
キャラクターの要素はユーザーが以下から選択する。
\begin{itemize}
  \item ヲタク
  \item JK
  \item 熱血
  \item 冷静
  \item おねぇ
  \item 一般人
\end{itemize}

\subsubsection{Gmailアカウント}
アカウントをコンテキストに含めることにより、
ユーザーごとのパーソナライズされた単語を推薦しやすくする。
またこのコンテキストは極めてプライバシー性が高いため、
デフォルトの設定では取得しないようにしている。

\subsection{動的コンテキスト}
\label{dynamiccontext}
以下動的コンテキストとして利用しているものについて述べる。

\subsubsection{時間}
時間をコンテキストに含めることにより、
時間ごとに入力されやすい単語を推薦しやすくする。
現状では時間と分を取得し、秒以下の数値は切り捨てている。

\subsubsection{日付}
日付をコンテキストに含めることにより、
日付ごとに入力されやすい単語を推薦しやすくする。

\subsubsection{位置情報}
位置情報をコンテキストに含めることにより、
場所ごとに入力されやすい単語を推薦しやすくする。

\subsubsection{加速度}
加速度をコンテキストに含めることにより、
加速度ごとに入力されやすい単語を推薦しやすくする。

\subsubsection{現在の入力}
現在未変換状態の文字が何であるかをコンテキストに含めることにより、
どのような単語が求められているかを推測し推薦しやすくする。

\subsubsection{IMEを起動してからの全ての入力}
IMEを起動してからの全ての入力をコンテキストに含めることにより、
共起表現のような入力単語と同じ文章内で使われやすい
単語を推薦しやすくする。

\subsubsection{入力の状態}
IMEを起動したタイミングなのか、
文字を入力中なのか、
変換または入力を確定したタイミングなのかという
三つの状態のうちどの状態であるかを取得する。
入力の状態をコンテキストに含めることにより、
アプリケーションに入力を開始した直後に入力しやすい単語や、
文章の区切りにおいて入力しやすい単語などを
推薦しやすくする。

\section{Jubatus}
\label{sec:jubatus}

\subsection{概要}
Jubatus\cite{jubatus}とは株式会社Preferred Infrastructureと
NTTソフトウェアイノベーションセンタが共同開発した、
オンライン機械学習向け分散処理フレームワークである。
本システムを使用することで推薦単語の候補計算をしている。
JubatusをRive日本語入力のシステムに組み込んだ理由は、
高速かつスケーラビリティに優れていたためである。
\cite{岡野原大輔:2013-01-01}
Rive日本語入力ではコンテキストを常に計算するため高速な処理が必要となり、
Jubatusを採用した。

\subsection{アルゴリズム}
取得した単語一つ一つがベクトルとなるような
多次元の特徴ベクトルを作っている。
そのn次元上で取得したコンテキストから新たなベクトルを作り、
近いものから順にスコア付けし、
推薦候補単語として返すようになっている。
機械学習アルゴリズムには
Adapting Regularization of Weight Vectors\cite{AROW}を使っている。

\section{推薦例}
本システムにおいて起こる推薦の具体的事例について説明する。
\begin{itemize}
  \item SFCにはじめてきた人がSFCならではの話題をつぶやく場合\mbox{}\\
    湘南台駅という過去の入力とSFCにいるという
    位置情報コンテキストを使うことで、
    「かもる」「諭吉像」「SFC」「ツインライナー」という推薦候補
    を得る。
  \item お盆休みに箱根に旅行にきた人が友人へ状況を説明する場合\mbox{}\\
    お盆休みという日付情報と箱根という位置情報コンテキストを使うことで、
    「箱根」「強羅」「温泉」という推薦候補単語を得る。
  \item サッカーを見ている時に日本が点を決めたことを友人に報告する場合\mbox{}\\
    LINE\footnote{http://line.me/}を開いているというコンテキストと
    インターネットにおいて最近盛り上がっている単語を使うことで、
    「ゴール」「日本」「先制」という推薦候補単語を得る。
  \item 朝起きて教授に遅刻の連絡をする場合\mbox{}\\
    メールクライントを開いているというコンテキストと
    起動時であるというコンテキストと
    自宅であるというコンテキストと
    昼という時間のコンテキストを使うことで、
    「申し訳ありません」「遅刻させていただきます」「よろしくお願いいたします」
    という推薦候補単語を得る。
\end{itemize}
