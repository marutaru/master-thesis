\chapter{推薦システム}
\label{chap:recommend}
本章では
推薦システムについて述べる。

\newpage
\section{概要}
ユーザーのコンテキストと入力を受け取り、
過去の全ての人の入力情報と比べることによって
ユーザーの入力する単語を予測し推薦するシステム。
クライアントを起動した瞬間から、
一つの操作(キーボードにタッチすることや候補単語へのタッチすること)
ごとに取得できる全てのコンテキストを入力と紐付けて
サーバーへ送信して、推薦システムがそのコンテキストにあった
推薦候補単語をクライアントへ送信するという
プロセスを繰り返すことで成り立っている。

\section{取得コンテキスト}
\label{sec:getcontext}
入力が開始されてから終了まで変わらないコンテキスト何度も取得すると
システムに負荷がかかるため、変わるものとそうでないものに分類した。
入力が開始してから変化しないコンテキストを静的コンテキスト。
入力が開始してから変化する可能性があるコンテキストを動的コンテキストとした。
コンテキストにはユーザーが設定するものと、
Rive日本語入力が自動的に取得するものがある。

\subsection{静的コンテキスト}
\label{staticcontext}
以下静的コンテキストとして利用しているものを述べる。

\subsubsection{アクテビティー}
\label{activity}
アクテビティーとは入力する対象のことである。
メールクライアントであったり、WEBブラウザアプリなどである。
アクテビティーをコンテキストに含めることにより、
サービスごとに入力されやすい単語を推薦されやすくなる。

\subsubsection{性別}
性別をコンテキストに含めることにより、
同じ性別の人が使っている単語を推薦されやすくなる。
要素はユーザーが以下から選択する。
\begin{itemize}
  \item 男性
  \item 女性
  \item その他
\end{itemize}

\subsubsection{年齢}
年齢をコンテキストに含めることにより、
同じ年齢の人が使っている単語を推薦されやすくなる。

\subsubsection{キャラクター}
キャラクターをコンテキストに含めることにより、
簡易なユーザークラスタリングをし、同じクラスタの人が使っている
単語を推薦されやすくなる。
要素はユーザーが以下から選択する。
\begin{itemize}
  \item ヲタク
  \item JK
  \item 熱血
  \item 冷静
  \item おねぇ
  \item 一般人
\end{itemize}

\subsubsection{Gmailアカウント}
アカウントをコンテキストに含めることにより、
ユーザーごとのパーソナライズされた単語を推薦されやすくなる。
またこのコンテキストは極めてプライバシー性が高いため、
デフォルトの設定では送信しないようになっている。

\subsection{動的コンテキスト}
\label{dynamiccontext}
以下利用している動的コンテキストについて述べる。

\subsubsection{時間}
時間をコンテキストに含めることにより、
時間ごとに入力されやすい単語を推薦されやすくする。

\subsubsection{日付}
日付をコンテキストに含めることにより、
日付ごとに入力されやすい単語を推薦されやすくする。

\subsubsection{位置情報}
位置情報をコンテキストに含めることにより、
場所ごとに入力されやすい単語を推薦されやすくする。

\subsubsection{加速度}
加速度をコンテキストに含めることにより、
加速度ごとに入力されやすい単語を推薦されやすくする。

\subsubsection{現在の入力}
現在未変換状態の文字が何であるかをコンテキストに含めることにより、
どのような単語が求められているかを推測し推薦する。

\subsubsection{IMEを起動してからの全ての入力}
IMEを起動してからの全ての入力をコンテキストに含めることにより、
入力単語と同じ文章内で使われやすい単語を推薦されやすくなる。

\subsubsection{入力の状態}
IMEを起動したタイミングなのか、
文字を入力中なのか、
変換または入力を確定したタイミングなのかという
三つの状態のどの状態であるかを取得する。
入力の状態をコンテキストに含めることにより、
最初に入力されやすい単語や、終わりに入力されやすい単語などを
推薦されやすくなる。

\section{Jubatus}
\label{sec:jubatus}

\subsection{概要}
Jubatus\cite{jubatus}とは株式会社Preferred Infrastructureと
NTTソフトウェアイノベーションセンタが共同開発した、
オンライン機械学習向け分散処理フレームワークである。
本システムを使用することで推薦単語の候補計算をしている。
JubatusをRive日本語入力のシステムに組み込んだ理由は、
高速かつスケーラビリティに優れていたためである。
\cite{岡野原大輔:2013-01-01}
Riveではコンテキストを常に計算するため高速な処理が必要となり、
採用した。

\subsection{アルゴリズム}
取得した単語一つ一つがベクトルとなるような
多次元の特徴ベクトルを作っている。
そのn次元上で取得したコンテキストから新たなベクトルを作り、
近いものから順にスコア付し、候補単語として返すようになっている。
機械学習アルゴリズムには
Adapting Regularization of Weight Vectors\cite{AROW}を使っている。

\section{推薦例}
本システムにおいて起こる推薦の事例について説明する。
\begin{itemize}
  \item SFCにはじめてきた人がSFCならではの話題をつぶやく場合\mbox{}\\
    湘南台駅という過去の入力とSFCにいるという
    位置情報コンテキストを使うことで、
    「かもる」「諭吉像」「SFC」「ツインライナー」という推薦候補
    を得る。
  \item お盆休みに箱根に旅行にきた人が友人へ状況を説明する場合\mbox{}\\
    お盆休みという日付情報と箱根という位置情報コンテキストを使うことで、
    「箱根」「強羅」「温泉」という推薦候補単語を得る。
  \item サッカーを見ている時に日本が点を決めたことを友人に報告する場合\mbox{}\\
    LINE\footnote{http://line.me/}を開いているというコンテキストと
    インターネットにおいて最近盛り上がっている単語を使うことで、
    「ゴール」「日本」「先制」という推薦候補単語を得る。
  \item 朝起きて教授に遅刻の連絡をする場合\mbox{}\\
    メールクライントを開いているというコンテキストと
    起動時であるというコンテキストと
    自宅であるというコンテキストと
    昼という時間のコンテキストを使うことで、
    「申し訳ありません」「遅刻させていただきます」「よろしくお願いいたします」
    という推薦候補単語を得る。
\end{itemize}
