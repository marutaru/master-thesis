\chapter{実装}
\label{implementation}

\section{Rive Client}
\subsection{システムフロー}

このシステムが立ち上がるとまず始めにonCreate()が呼ばれ、初期化を行う。
その後onStartInput()メソッドが呼ばれコンテキストを取得する。
このコンテキストはデバイスの情報を取得する。
コンテキストを取得し次第、候補単語を取得する。
この候補単語はサーバーと通信した上で取得する。
これをユーザーが一つの操作を行うたびに繰り返す。
ここで言う一つの動作とは、キーボード上の一つの文字を押すことや、
候補の単語をタップすること、
あるいは文字をデリートすることも含まれる。
最終的にユーザーが入力を終了した場合にはonFinishInput()が呼ばれ、
今回のユーザが行った動作とコンテキストを紐付けサーバーに送信する。
通信が終わり次第onDestory()が呼ばれ本システムは終了する。

\begin{figure}[htbp]
  \begin{center}
    \includegraphics[width=14cm,bb=0 0 469 366]{images/clientflow}
  \end{center}
  \caption{Rive Clientフローイメージ}
  \label{fig:clientflow}
\end{figure}

\subsection{コンテキスト取得}

取得するコンテキストについては\ref{sec:getcontext}項を参照。
コンテキストはユーザーに入力してもらえるものは入力してもらう。
またデバイスで取得可能なものを全て取得し、推薦システムに使う。

\section{Rive Server}
\subsection{システム構成}

サーバーの中身は大きくRoutting Server,beforeRules,afterRules,Jubatusの
4つを実装した。
それぞれ候補単語の計算手法が異なるため此のような分割になっている。
\begin{figure}[htbp]
  \begin{center}
    \includegraphics[width=14cm,bb=0 0 466 316]{images/riveserver.png}
  \end{center}
  \caption{Rive Server概要図}
  \label{fig:riveserver}
\end{figure}

\subsection{Routing Server}
このサーバーはふたとおりの役割を持っている。
一つはデバイスから受け取ったコンテキストデータを
beforeRules,Jubatus,afterRulesの3つの計算エンジンに振り分け送信する役割である。
もうひとつは受け取ったデータを受け取り次第、デバイスに送信する役割である。

\subsection{beforeRules}
このエンジンはコンテキストデータが送られてきた際に、
一定のルールに基づいているものをまとめている。
このルールは著者?開発者?がよく使われる単語を推測し実装した。
例えば、Twitter\cite{Twitter}クライアントに入力を行っている(行おうとしている)
場合に「なう」「@」の候補単語を返すようになっている。

\subsection{Jubatus}
このシステムについては第\ref{sec:jubatus}項参照。

\subsection{afterRules}

\section{Rive Analytics}

\section{Rive BatchProcessing}

\section{Rive WebService}
