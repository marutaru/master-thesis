\chapter{序論}
\label{chap:introduction}
本章は序論である。

\newpage
\section{研究の目的}
ユーザの文字入力の速度を向上させる。
ユーザの文字入力における楽しさを向上させる。
\section{用語定義}

本論文において使用する用語を以下のように定義する。

\begin{description}
  \item[デバイス]\mbox{}\\
    本論文に置いてデバイスとは単にハードウェアを意味している。
  \item[モバイルデバイス]\mbox{}\\
    本論文において、モバイルデバイスとは持ち運び可能であり、
    インターネットに接続可能な端末のことを意味している。
  \item[スマートフォン]\mbox{}\\
    本論文において、スマートフォンとは多機能なモバイルデバイスであり、
    パソコンとしての機能を果たせるものを意味している。
    具体的にはAndroid,iOSなどを搭載しているものである。
  \item[タブレット]\mbox{}\\
    本論文において、スマートフォンと同じ要件を満たしながら、
    画面のサイズがスマートフォンより大きい物を意味している。
  \item[IME]\mbox{}\\
    IMEとはInput Method Editorの略である。
    コンピュータなどの情報機器において文字入力をする際に、
    それを手助けするシステムのことをIMEと呼ぶ。
    \begin{quote}
      インプット メソッド エディタ(Input Method Editor)、
      訳して入力方式エディタとは、パーソナルコンピュータを
      はじめとした情報機器で、文字入力を補助する
      ソフトウェアである。略称IME。
      通常は言語ごとに別個に用意されるため、
      言語に応じ日本語インプットメソッドエディタ
      (日本語IME、日本語文字入力補助ソフト)
      などと呼ばれる。\cite[出典]{ime}
    \end{quote}
  \item[コンテキスト]\mbox{}\\
    コンテキストとは本論文において状況という意味で使っている。
  \item[予測変換]\mbox{}\\
    予測変換とはIMEにおける予測変換のことを示している。
  \item[省入力]\mbox{}\\
    本論文において省入力とは
    入力の時間または手間を省くことを示している。
  \item[候補単語]\mbox{}\\
    本論文に置いて候補単語とは入力における、
    変換候補と推薦候補の二つを合わせた表現である。
\end{description}

\section{本論文の構成}


第\ref{chap:background}章。

第\ref{chap:design}章。

第\ref{chap:recommend}章。

第\ref{chap:implementation}章。

第\ref{chap:userinterface}章。

第\ref{chap:discussion}章。

第\ref{chap:related}章。

第\ref{chap:conclusion}章。
