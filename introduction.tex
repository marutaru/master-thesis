\chapter{序論}
\label{chap:introduction}
本章では研究の目的と論文の構成について述べる。

\newpage
\section{研究の動機}
昨今、インターネット環境の充実化と共に
一般の人々は日本語の入力をいつでもどこでも行うようになった。
いままでならば文字入力を滅多に行わなかった人々なども
facebook\footnote{http://www.facebook.com/}
やtwitter\footnote{http://twitter.com/}
といったSNS\footnote{Social Networking Serviceの略称}や
LINE\footnote{http://line.me/}等
のコミュニケーションアプリの普及と共に
入力を頻繁に行っている。
しかし日本語IMEはできることがコンピュータにおけるIMEと変わらず
モバイルデバイスの性能を最大限活かしているとは言いづらい。
そこでモバイルデバイスの性能を最大限活かした、
今までとは違った入力体験をしたいと考え、
新しい日本語入力IMEの設計・開発を行った。

\section{研究の目的}
本研究の目的は世の中の日本語を入力する人々が
モバイルデバイスへの入力を行う際に、
ユーザの文字入力における面白さを向上させると共に
文字入力の速度を向上させるシステムの
実装と提案をすることである。
その目的を達成するために
「Rive日本語入力」というシステムを実装・開発した。
本論文においてモバイルデバイスとは
持ち運び可能でありインターネットに接続可能な
端末のことを示していて、
何時でもどこでも入力を行うということが前提となっている。

%\section{用語定義}
%本論文において使用する用語を以下のように定義する。
%\begin{description}
%  \item[デバイス]\mbox{}\\
%    本論文に置いてデバイスとは単にハードウェアを意味している。
%  \item[モバイルデバイス]\mbox{}\\
%    本論文において、モバイルデバイスとは持ち運び可能であり、
%    インターネットに接続可能な端末のことを意味している。
%  \item[スマートフォン]\mbox{}\\
%    本論文において、スマートフォンとは多機能なモバイルデバイスであり、
%    パソコンとしての機能を果たせるものを意味している。
%    具体的にはAndroid,iOSなどを搭載しているものである。
%  \item[タブレット]\mbox{}\\
%    本論文において、スマートフォンと同じ要件を満たしながら、
%    画面のサイズがスマートフォンより大きい物を意味している。
%  \item[IME]\mbox{}\\
%    IMEとはInput Method Editorの略であり
%    文字入力補助ソフトと言われる。。
%    コンピュータなどの情報機器において文字入力をする際に、
%    それを手助けするシステムのことをIMEと呼ぶ。
%  \item[コンテキスト]\mbox{}\\
%    コンテキストとは本論文において状況という意味で使っている。
%  \item[集合知]\mbox{}\\
%    集合知とは複数のデータを集める事によって
%    新たな価値を出すという意味で使っている。
%  \item[省入力]\mbox{}\\
%    本論文において省入力とは
%    入力の時間または手間を省くことを示している。
%  \item[候補単語]\mbox{}\\
%    本論文に置いて候補単語とは入力における、
%    変換候補と推薦候補の二つを合わせた表現である。
%  \item[クライアント]\mbox{}\\
%    本論文においてクライアントとは
%    ユーザーが直接利用するアプリケーション
%    あるいはそのデバイスを指す表現である。
%\end{description}

\section{本論文の構成}
本論文は全\ref{chap:publication}章で構成される。\mbox{}\\
第\ref{chap:background}章ではRive日本語入力の開発に至った背景について述べる。\mbox{}\\
%第\ref{chap:design}章ではRive日本語入力の設計について述べる。
第\ref{chap:recommend}章では推薦システムの仕様について述べる。\mbox{}\\
第\ref{chap:implementation}章ではRive日本語入力の設計と実装について述べる。\mbox{}\\
第\ref{chap:userinterface}章ではRive日本語入力のユーザインタフェースについて述べる。\mbox{}\\
第\ref{chap:discussion}章ではRive日本語入力の課題について述べる。\mbox{}\\
第\ref{chap:related}章では関連研究を述べる。\mbox{}\\
第\ref{chap:conclusion}章では本研究の結論を述べる。\mbox{}\\
最後に第\ref{chap:publication}章では研究に関する発表について述べる。\mbox{}\\
