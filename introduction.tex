\chapter{序論}
\label{chap:introduction}
本章では研究の目的、使用する用語の定義と
論文の構成について述べる。

\newpage
\section{研究の目的}
本研究の目的は世の中の日本語を入力する人々が
コンピュータあるいはモバイルデバイスへの入力を行う際に、
ユーザの文字入力における楽しさを向上させると共に
文字入力の速度を向上させることが目的である。
その目的を達成するためにRive日本語入力というシステムを実装・開発した。

\section{用語定義}
本論文において使用する用語を以下のように定義する。

\begin{description}
  \item[デバイス]\mbox{}\\
    本論文に置いてデバイスとは単にハードウェアを意味している。
  \item[モバイルデバイス]\mbox{}\\
    本論文において、モバイルデバイスとは持ち運び可能であり、
    インターネットに接続可能な端末のことを意味している。
  \item[スマートフォン]\mbox{}\\
    本論文において、スマートフォンとは多機能なモバイルデバイスであり、
    パソコンとしての機能を果たせるものを意味している。
    具体的にはAndroid,iOSなどを搭載しているものである。
  \item[タブレット]\mbox{}\\
    本論文において、スマートフォンと同じ要件を満たしながら、
    画面のサイズがスマートフォンより大きい物を意味している。
  \item[IME]\mbox{}\\
    IMEとはInput Method Editorの略であり
    文字入力補助ソフトと言われる。。
    コンピュータなどの情報機器において文字入力をする際に、
    それを手助けするシステムのことをIMEと呼ぶ。
  \item[コンテキスト]\mbox{}\\
    コンテキストとは本論文において状況という意味で使っている。
  \item[集合知]\mbox{}\\
    集合知とは複数のデータを集める事によって
    新たな価値を出すという意味で使っている。
  \item[省入力]\mbox{}\\
    本論文において省入力とは
    入力の時間または手間を省くことを示している。
  \item[候補単語]\mbox{}\\
    本論文に置いて候補単語とは入力における、
    変換候補と推薦候補の二つを合わせた表現である。
  \item[クライアント]\mbox{}\\
    本論文においてクライアントとは
    ユーザーが直接利用するアプリケーション
    あるいはそのデバイスを指す表現である。
\end{description}

\section{本論文の構成}
第\ref{chap:background}章ではRive日本語入力の開発に至った背景について述べる。
第\ref{chap:design}章ではRive日本語入力の設計について述べる。
第\ref{chap:recommend}章では推薦システムの使用について述べる。
第\ref{chap:implementation}章ではRive日本語入力の実装について述べる。
第\ref{chap:userinterface}章ではRive日本語入力のユーザーインターフェースについて述べる。
第\ref{chap:discussion}章ではRive日本語入力の課題について述べる。
第\ref{chap:related}章では関連研究を述べる。
第\ref{chap:conclusion}章では本研究の結論を述べる。
最後に第\ref{chap:publication}章では研究に関する発表について述べる。
