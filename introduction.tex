\chapter{序論}
\label{chap:introduction}

\section{研究の目的}
ユーザの文字入力の速度を向上させる。
ユーザの文字入力における楽しさを向上させる。
\section{用語定義}

本論文において使用する用語を以下のように定義する。

\begin{description}
  \item [デバイス]
    本論文に置いてデバイスとは単にハードウェアを意味している。
  \item [モバイルデバイス]
    本論文において、モバイルデバイスとは持ち運び可能であり、
    インターネットに接続可能な端末のことを意味している。
  \item [スマートフォン]
    本論文において、スマートフォンとは多機能なモバイルデバイスであり、
    パソコンとしての機能を果たせるものを意味している。
    具体的にはAndroid,iOSなどを搭載しているものである。
  \item [タブレット(端末)]
    本論文において、スマートフォンと同じ要件を満たしながら、
    画面のサイズが大きい物を意味している。
  \item [IME]
    \begin{quote}
      インプット メソッド エディタ(Input Method Editor)、
      訳して入力方式エディタとは、パーソナルコンピュータを
      はじめとした情報機器で、文字入力を補助する
      ソフトウェアである。略称IME。
      通常は言語ごとに別個に用意されるため、
      言語に応じ日本語インプットメソッドエディタ
      (日本語IME、日本語文字入力補助ソフト)
      などと呼ばれる。\cite[出典]{ime}
    \end{quote}
  \item [コンテキスト]
    コンテキストとは本論文において状況として使われている。
  \item [予測変換]

\end{description}

\section{本論文の構成}


第\ref{chap:background}章。

第\ref{chap:design}章。

第\ref{chap:application}章。

第\ref{chap:discussion}章。

第\ref{chap:related}章。

第\ref{chap:conclusion}章。
