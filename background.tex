\chapter{背景}
\label{chap:background}
本章では本研究の背景となった、
モバイルデバイスへの文字入力機会増加、入力のシステム歴史、文字入力への問題
について述べる。

\newpage
\section{モバイルデバイスへの文字入力の機会の増加}
モバイルデバイスにおいて、文字入力をする機会がとても多い。

\subsection{モバイルデバイスの普及}
現在、日本において
スマートフォンやタブレットなどのモバイルデバイスが大幅かつ急激に普及した。
総務省による「平成25年通信利用動向調査」
\cite{communicationreport}の
「通信端末世帯保有率の推移」
(図:\ref{fig:mobiledevicespread})によると、
\begin{figure}[htbp]
  \begin{center}
    \includegraphics[width=160mm,bb=0 0 856 494]{images/mobiledevicespread.png}
    \caption{情報通信端末世帯保有率の推移(出典:\cite{communicationreport})}
    \label{fig:mobiledevicespread}
  \end{center}
\end{figure}
平成22年には9.7%しかなかったスマートフォンの普及率は、
平成25年には62.6%と急激に成長している。
平成22年から平成25年の3年間で52.9%の伸びを見せており、
これはパソコンの保有率が最も伸びた平成11年から平成14年の
伸び率を大きく上回る数字となっている。
このデータからもスマートフォンの普及は未だかつてないほどの
速度で進んでいることがわかる。

\subsection{モバイルデバイスの用途}
スマートフォンはパソコンの様に多くのことができるが、
文字入力を行う可能性が高いアプリケーションは頻繁に使用されている。
リサーチバンクによるアンケート調査\cite{researchbanksmartphone}
(図:\ref{fig:purpose})によると
\begin{figure}[htbp]
  \begin{center}
    \includegraphics[width=115mm,bb=0 0 500 1001]{images/purpose.png}
    \caption{スマートフォンでどのような機能・アプリを使っているか(出典:\cite{researchbanksmartphone})}
    \label{fig:purpose}
  \end{center}
\end{figure}
メールアプリを使う人が78.1%、SNSアプリを使う人が44.8%、
コミュニケーションアプリを使う人が44.5%となっている。
メールアプリ、コミュニケーションアプリ、SNSアプリはそれぞれ
情報の受信端末としてだけ使う場合には文字入力の必要はないが、
情報を発信しようとする場合には文字入力が必要である。
これら以外の他のアプリにおいても適時文字入力が必要である。
つまりスマートフォンを使うユーザーは文字入力の機会がとても多いことがわかる。

\section{入力システムの歴史}
現在どのようなOSを使用していても必ずIMEアプリケーションは搭載されている。
これらのモバイルデバイスを使う上で日本語の入力は欠かすことができない操作である。
デバイスの性能は携帯電話の頃から劇的に向上している。
パソコンに遜色ないようなメモリやCPUを積んでいるものも多く市販されている。
しかし文字入力に関してはあまり成長していない。

文字入力にもデバイスに適したよりよいものがある。
文字入力のユーザの体験ももっと向上すべきである。
今回は日本人向けのシステムとして作った。

ユーザにはそれぞれコンテキストがある。
コンテキストによって入力したい単語は推測できるのではないか

\subsection{文字入力への機械増加}

\section{文字入力の歴史}

\section{問題点と期待}
楽天リサーチによる「スマートフォンに関する調査」
(出典:\cite{rakutensmartphone})によると、
スマートフォンを使うユーザーが挙げる不満点の中で
最も割合が大きいのが「文字入力がしにくい」ということである。
(出典:\cite{rakutensmartphone})
\begin{figure}[htbp]
  \begin{center}
    \includegraphics[width=140mm,bb=0 0 589 368]{images/dissatisfaction.png}
    \caption{スマートフォンを利用している際に不便だと感じる点(出典:\cite{rakutensmartphone})}
    \label{fig:dissatisfaction}
  \end{center}
\end{figure}

利用時間長すぎ
スマートウォッチやグーグルグラスなどでは文字入力めんどい
