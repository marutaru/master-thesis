% ■ アブストラクトの出力 ■
%	◆書式:
%		begin{jabstract}〜end{jabstract}	:日本語のアブストラクト
%		begin{eabstract}〜end{eabstract}	:英語のアブストラクト
%		※ 不要ならばコマンドごと消せば出力されない。



% 日本語のアブストラクト
\begin{jabstract}

モバイルデバイスの進化と普及と共に文字入力の機会は増加している。
しかし現状スマートフォンを利用してる人々の
日本語入力IMEには大きな変化はみられない。
本研究では現状のモバイルデバイスに合った、
より楽しく早い入力を可能にする日本語入力IME
「Rive日本語入力」を設計し実装した。
本システムにおける最大の特徴はコンテキストによる
集合知を使った候補単語の推薦を行うことにある。
コンテキストを使用することによって、
ユーザは意識することなくその状況に
適した入力候補単語を受け取り、
選択することで入力を完了する。
開発した推薦エンジンの設計や実装を述べると共に
採用したユーザインタフェースを提示することで、
既存のIMEにはない有用性を述べる。
また本システムにおける課題と
それに対するアプローチを述べることで、
日本語入力における本研究の意義を論ずる。

\end{jabstract}

% 英語のアブストラクト
\begin{eabstract}

We are increasing at an opportunity to input character
with evolution and the spread of mobile devices.
But smartphone user's inputting hardly changes with Japanese IME.
In this research, I designed and implemented Japanese IME:
"RiveJapaneseInput" which enables users
to quick input and interesting input.
The most feature in this system is recommend words from
user's context.
User can receive recommend candidate words suitable for
the situation unconsciously.
And he only chooses these words.
I propose some interfaces in this system which can apply
other systems.
In this paper, I implemented Japanese IME
and discuss advantage of this system.

\end{eabstract}
