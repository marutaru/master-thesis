% ■ アブストラクトの出力 ■
%	◆書式:
%		begin{jabstract}〜end{jabstract}	:日本語のアブストラクト
%		begin{eabstract}〜end{eabstract}	:英語のアブストラクト
%		※ 不要ならばコマンドごと消せば出力されない。



% 日本語のアブストラクト
\begin{jabstract}

モバイルデバイスの進化と普及と共に文字入力の機会は増加している。
しかし現状スマートフォンを利用してる人々の
日本語入力IMEには大きな変化はみられない。
本研究では現状のモバイルデバイスに適した、
より早く面白い入力を可能にする日本語入力IME
「Rive日本語入力」を設計し実装した。
本システムにおける最大の特徴はコンテキストを用いた
候補単語の推薦を行うことである。
コンテキストと入力を関連付けると共に、
過去の全てのユーザーの入力データを利用することで
ユーザーの入力したいであろう単語を推測する。
ユーザーは意識することなく推薦候補単語を受け取り、
選択するだけで入力をすることができる。
本論文においては
開発した日本語入力IMEの設計や実装、
採用したユーザインタフェースを提示することで、
システムの有用性を述べる。
また本システムの課題やそれに対するアプローチを述べることで、
日本語入力IMEにおける本研究の意義を論ずる。

\end{jabstract}

% 英語のアブストラクト
\begin{eabstract}

%We are increasing at an opportunity to input text
%with evolution and the spread of mobile devices.
%But smartphone user's inputting hardly changes in Japanese IME.
%In this research, I designed and implemented Japanese IME:
%"Rive Japanese Input" which enables the quick input
%and interesting input of the user.
%The biggest feature of this system is to recommend candidate words
%from the context.
%I connect input data with context and suppose the text
%that the user want to input by using the past input data
%of all users
%The user receives recommendation candidate words unconsciously.
%And he complete input only to choose words in recommendation
%candidate words.
%In this paper, I propose usability of this system by adopting
%user interface and the implementation of Japanese IME which
%I developed.
%And I discuss importance of this study.
Text input on mobile devices like smartphones
are still difficult, especially when entering Japanese texts where
special software is required for converting keyboard input to
Japanese texts including thousands of characters.
We have developed a new context-based Japanese input system called
"Rive" that can help users entering texts by predicting the next
input word from the user's contexts including the state of
the application, location of the user, the age of the user,
etc.
Using Rive, we can dramatically reduce the pain of entering Japanese
text, since the system can predict the next input word
correctly in many cases.
In this paper, we describe the design and implementation of Rive,
and show our experience of Rive after using it every day
for more than one year.

\end{eabstract}
