\chapter{関連研究}
\label{chap:related}

\section{HumanComputation}
コンピュータの計算能力だけでは解決できない問題を、人間の処理能力を計算資源として利用することによって解決する手法は、
ヒューマンコンピュテーション\cite{humancomputation}と呼ばれ、様々な研究が行なわれている。

\cite{recaptcha}
\cite{vizwiz}

\section{CrowdSourcing}
インターネットを介して、不特定多数の人に仕事を依頼する仕組みとして、クラウドソーシングというものがある。
クラウドソーシングとは、アウトソーシングという言葉を改変した造語である\cite{riseofcrowdsourcing}。
安価かつ必要なだけの人員をすぐに利用することができる。
近年、この分野の研究も大きな注目を浴び、様々な研究が行なわれている。

クラウドソーシングのためのプラットフォームとしては、Amazon's Mechanical Turkが有名である\cite{amt}。
クラウドソーシング関連の多くが、MechanicalTurkを利用する。
Turkit\cite{turkit}は、このMechanical Turkを簡単に利用するためのツールキットだ。
「投票」等のクラウドソーシングにおいて頻繁に依頼される処理と
それに伴うアルゴリズムを簡単に実行することができる。
また、クラウドソーシングした処理の結果を保存しておくことで、プログラムの実行に失敗しても、
再度クラウドソーシングに処理依頼をするのではなく、保存しておいた結果を元にプログラムを再度実行できる。

Barowyらは、Automanというプログラミング言語Scala上で動作するDomain Specific Languageを提案した\cite{automan}。
また、CrowdProgrammingという概念を提唱している。
Automanは、プログラミング言語内において同様の記法によって、
クラウドソーシングによる計算とコンピュータによる計算を統合した。
また、クラウドソーシングのための品質管理やスケジューリングのための機能等も持つ。

Franklinらは、CrowdDBというSQLを拡張した。

不明瞭であったり、不完全なデータの処理や、抽象的な比較
\cite{crowddb}

\cite{cylog}
\cite{crowdforge}
\cite{community-based-crowdsourcing}
\cite{soylent}

クラウドソーシング分野の研究では、インターネットを介して不特定多数の人を計算資源として利用することによって、
コンピュータのみでは実現できなかったような処理や、より大規模な人力処理を実現させている。
本研究では、不特定多数の人ではなく、特定可能な人を対象としたものである。

% \section{Programming Environments}
\section{Social Computing}
\cite{dog},
\cite{jabberwocky},
\cite{personal-api},
\cite{social-machines},

\section{Human as Sensor}
\cite{prism}
\cite{moboq}
\cite{medusa}

\section{その他}
\cite{hapticturk}
\cite{sharedo}
