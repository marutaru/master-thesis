\chapter{関連研究}
\label{chap:related}
本章ではRive日本語入力に関連した研究について述べる。

\newpage
\section{IMEに関する研究・製品}
Rive日本語入力と既存の研究や製品を比較することで、
Rive日本語入力の特色を述べる。

\subsection{コンテキストアウェアIMEの実現へ向けた動的辞書生成手法の提案}
荒川らが研究・提案した動的辞書生成手法\cite{dynamicdictionarygeneration}
である。
ユーザのコンテキストに応じて適切な単語を
推薦することによって、
携帯端末の文字入力を改善する
コンテキストアウェアIMEの概念を提案し、
そのために必要な動的辞書生成手法に提示している研究である。
コンテキストを利用し、
様々な人のデータをひもづけることによって一つの辞書を作るという点に置いて
Rive日本語入力と類似している。
差異はコンテキストのリアルタイム性にある。
Rive日本語入力においてコンテキストは
動的コンテキストと静的コンテキスト
に分かれているが、この動的辞書生成においては静的コンテキスト
として定義した位置情報についてのみ
考察しているがRive日本語入力では、
複数のコンテキストを総合的に解析する手法について設計し開発した。

\subsection{Social IME}
奥野が開発したユーザ間で辞書を共有するIME\cite{socialime}
である。
Rive日本語入力において複数のユーザが一つの辞書を利用するという
集合知を用いた辞書の発想は類似している。
このIMEにおいて使われているシステム\cite{奥野陽:2009-03-18}では
以前の入力などと言ったコンテキストが考慮されおらず、
Rive日本語入力ではそれらをコンテキストに導入することで、
文章全体における共起表現などの推薦を可能にした。

\subsection{iWnn}
iWnnとは\cite{iwnn}
omron社が開発した日本語IMEであり、
現在AndroidOSにおいて多くの機種で標準搭載しているものである。
携帯電話向けに実装されていて、
時間情報やメールの送信相手の情報からコンテキストを推測し、
適切な推薦を行う点に置いて、Rive日本語入力と類似している。
コンテキストについては開発者の決めたものから推測するため、
未知のコンテキストなど柔軟なコンテキスト推測ができないところが、
Rive日本語入力システムと異なる。

\subsection{Google日本語入力}
Google日本語入力はGoogleによって開発されたIMEである。
Googleの検索システムと連動しており最新かつ膨大な情報を使った
辞書が利用可能である。
検索システムを使っているため、
集合知を使いリアルタイムな情報を得ることができるという点に置いて
Rive日本語入力と類似している。
その上でリアルタイムな情報を推薦するためのアプローチが異なっている。
Google日本語入力ではコンテキストとして利用されるものが過去の入力
だけになっている。

%\subsection{各種クライアント}
%Rive日本語入力システムにおいて
%アクテビティー(\ref{staticcontext}項)を
%コンテキストに含めているが、
%ごとに特化型クライントが存在する場合、
%それらはIMEとしての一部機能を備えていることが多い。
%Twitterクライアントであれば、
%リプライを送る際に相手のアカウント名が自動で
%入力されるようになる等の入力補助がそれにあたる。

\section{コンテキストを使用した推薦システム}
コンテキストを利用したシステムに関する研究は始め、
位置情報や時間情報利用したシステムが多かった。
その後そこにユーザー情報を加えたシステムの研究が行われた。
更にその後には、他のコンテキストを加えたシステムの研究が発展してきた
\cite{okukenta}。

\subsection{コンテキストの分類}
Rive日本語入力においては第\ref{chap:recommend}章において
述べたように実装上、動的コンテキストと静的コンテキストの
2種類に分類したが、
コンテキストには様々な分類方法が存在する。
まずソフトウェア開発に関するコンテキストでは
Kai Petersonらによる研究で\cite{KAIPETERSON}
Product,Process,Practice/Tool,People,Organization,Market
という6つにコンテキストが分類されている。
しかしRive日本語入力において
想定しているコンテキストとは多少異なるため、
Rive日本語入力では
主にStrangらの研究\cite{contextsurvey}を参考にした。
その研究によるとコンテキストとは以下に分類される。
\begin{itemize}
  \item Key-Value Models
  \item Markup Schema Models
  \item Graphical Models
  \item Object Oriented Models
  \item Logic Based Models
  \item Ontology Based Models
\end{itemize}
Rive日本語入力においては
集合知を用いた機械学習手法により、
コンテキストを用いた推薦を行うため、
コンテキストを決める際の参考ではなく
特にbeforeRulesやafterRules等の閾値を考察する際の参考とした。

\subsection{コンテキストを用いた入力推薦システム}
ここではRive日本語入力を作る上で参考とした、
コンテキストを用いた推薦システムについて述べる。
特にRive日本語入力に近いと思われるものについて挙げる。
まずRohodesらによるシステム\cite{ROHODESB.J.:1997}
が挙げられる。
このシステムはウェアラブルデバイスを使ってコンテキストを取得し、
emacs\footnote{http://www.gnu.org/software/emacs/emacs.html}
における文字入力を助けるシステムである。
このシステムにおいては5秒ごとに
コンテキストを取得し、ユーザーにエディタにおける
最適な文字入力の提案を行うというシステムである。
エディタという限られた状況に絞っているためこのシステムは有用であるが、
IMEでは様々な状況で様々な入力を行うため、
予め開発者によって決められている候補だけでは
充分ではない。
Rive日本語入力においては過去の全てのユーザの入力
を使うことでこの問題を解決している。
また高橋らによる状況と行動に関する研究
\cite{高橋公海:2013-07-15}も挙げられる。
この研究ではコンテキストにおけるユーザの思考を行動という
形で表しているが、これはRive日本語入力における
候補単語を選択するという入力の方式に類似している。
また同様な研究にはグェンらによる研究
\cite{ミンテイグェン:2010-01-15}もある。
しかしこれらのシステムは限定的な状況における
コンテキストによる予測を行っており、
あらゆる状況における入力を想定しているRive日本語入力とは異なる
が限定的な状況においてより有用な推薦をするための
参考になる研究といえる。
