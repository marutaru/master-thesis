\chapter{関連研究}
\label{chap:related}
本章ではRive日本語入力に関連した研究について述べる。

\newpage
\section{IMEに関する研究・製品}
Rive日本語入力と既存の研究や製品を比較することで、
Rive日本語入力の特色を述べる。

\subsection{コンテキストアウェアIMEの実現へ向けた動的辞書生成手法の提案}
荒川らが研究・提案した動的辞書生成手法である。\cite{dynamicdictionarygeneration}
コンテキストを利用し、
様々な人のデータをひもづけることによって一つの辞書を作るという点に置いて
Rive日本語入力と類似している。
差異はコンテキストのリアルタイム性にある。
Rive日本語入力においてコンテキストは
動的コンテキストと静的コンテキスト\ref{recommmend}
に分かれているが、この動的辞書生成においては静的コンテキスト
として私が定義したものしか考えられていない。
またこのシステムではコンテキストについて位置情報についてのみ
考察しているがRive日本語入力では、
複数のコンテキストを総合的に解析する手法について設計し開発した。

\subsection{Social IME}
奥野が開発したユーザ間で辞書を共有するIMEである。\cite{socialime}
Rive日本語入力において複数のユーザが一つの辞書を利用するという
集合知を用いた辞書の発想は類似している。
このIMEにおいて使われているシステム\cite{奥野陽:2009-03-18}では
以前の入力などと言ったコンテキストが考慮されおらず、
Rive日本語入力ではそれらをコンテキストに導入することで、
文章全体における共起表現などの推薦を可能にした。

\subsection{iWnn}
omron社が開発した日本語IMEであり、\cite{iwnn}
現在AndroidOSにおいて多くの機種で標準搭載しているものである。
携帯電話向けに実装されていて、
時間情報やメールの送信相手の情報からコンテキストを推測し、
適切な推薦を行う点に置いて、Rive日本語入力と類似している。
コンテキストについては開発者の決めたものから推測するため、
未知のコンテキストなど柔軟なコンテキスト推測ができないところが、
Rive日本語入力システムと異なる。

\subsection{Google日本語入力}
Google日本語入力はGoogle社によって開発されたIMEである。
Googleの検索システムと連動しており最新かつ膨大な情報を使った
辞書が利用可能である。
検索システムを使っているため、
集合知を使いリアルタイムな情報を得たいという点に置いて
Rive日本語入力と類似している。
その上で実現するためのアプローチが異なっている。
Google日本語入力ではコンテキストとして利用されるものが過去の入力
だけになっている。

\subsection{各種クライアント}
Rive日本語入力システムにおいてコンテキスト推測で、
アクテビティー(第\ref{staticcontext})ごとに特化型クライントが存在する場合、
それらはIME機能を備えていることが多い。
Twitterクライアントであれば、
リプライを送る際に相手のアカウント名が自動で
入力されるようになる等の入力補助がある。

\section{コンテキストを使用した推薦システム}
コンテキストを利用したシステムに関する研究は始め、
位置情報や時間情報利用したシステムが多かった。
その後、そこにユーザー情報を加えたシステムの研究が行われた。
更にその後には、他のコンテキストを加えたシステムの研究が行われた。
\cite{okukenta}

\subsection{コンテキストの分類}
Rive日本語入力においては\ref{chap:recommend}において
述べたように実装上、動的コンテキストと静的コンテキストの
2種類に分類したが、
Strangらの研究\cite{contextsurvey}によると以下に分類される。
\begin{itemize}
  \item Key-Value Models
  \item Markup Schema Models
  \item Graphical Models
  \item Object Oriented Models
  \item Logic Based Models
  \item Ontology Based Models
\end{itemize}
しかしこの分類は結果を見すえた上での分類であり、
コンテキストがどのような推薦結果を提示するのか
予測が難しいRive日本語入力における
コンテキストの分類には適さなかった。

\subsection{}
