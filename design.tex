\chapter{設計}
\label{chap:design}

\section{Rive日本語入力system}

Rive日本語入力システムは以下のものによって構成されている

\begin{itemize}
  \item Rive Client
  \item Rive Server
  \item Rive Analytics
  \item Rive BtachProcessing
  \item Rive Webservice
\end{itemize}

これら5つのシステムがお互いに作用することで
Rive日本語入力システムが実現されている。


% 各要素について1行ぐらいで軽く説明する
% システム図みたいなので全体を説明する

\section{Rive Client}
\subsection{概要}
モバイルデバイスで利用可能なアプリケーション本体。
\subsection{実装}
AndroidOS向けのIMEとして実装した。

\section{Rive Server}
\subsection{概要}
Rive Clientと通信し、適切な候補単語を推測するシステム
\subsection{実装}


\section{Rive Analytics}
\subsection{概要}
文字入力のデータを受け取り解析することでシステムの向上に役立てる
\subsection{実装}
Rive Clientからデータを受け取りバージョンごとに管理する

\section{Rive BatchProcessing}
\subsection{概要}
定期的に処理を行うシステムの総称
\subsection{実装}
インターネット上のデータをクロールする

\section{Rive WebService}
\subsection{概要}
システムを試用するためのWEBページとして実装。
また本システムの紹介も兼ねている。
\subsection{実装}
WEBページとして実装した。

\section{システム間通信}
\subsection{実装}
システム間はWebsocketまたはhttpによって実現した。
