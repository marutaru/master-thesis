\chapter{課題}
\label{chap:discussion}
本章ではRive日本語入力における課題とそれに対する
アプローチについて述べる。

\newpage
\section{コールドスタート問題}
本システムは集合知を使ってレコメンドを行っているため、
データが充分に集まるまで適切な推薦を行うことができないという問題がある。
本来こういった集合知を使ったシステムにおいては、
外部システムやアプリケーション
から持ってきたデータ等をシステム用に加工し、
シードデータとして使うことでこの類の問題を解決する。
しかしRive日本語入力については一定のコンテキストが付加された
入力データが必要であるがインターネット上には存在しないデータである。
そこで第\ref{sec:riveserver}項において述べた、
beforeRulesとafterRulesの実装を行った。
これを実装したことによりデータが少ない状況でも
一定の推薦候補単語を得ることができるようになった。

\section{省入力化と多様性のジレンマ}
Rive日本語入力を使用し、キーボードを使うことなく
候補単語の選択だけで入力が
完了するようになると本システムの目的である省入力化は達成される。
しかし本システムを使うあらゆる人が同じ言葉あるいは単語を使う
ことになって日本語の多様性が失われてしまうのではないか
という懸念がある。
これが省入力化と多様性のジレンマである。
この懸念はコンテキストへの誤解から起こる問題であると考える。
物理的なコンテキストしか取得していないと確かに
この問題を解決することはできないが、
精神的なコンテキストを導入することによって解決すると考えている。
物理的なコンテキストとは気温や位置などの客観的に
測ることができるコンテキストのことであり、
精神的なコンテキストとはユーザーが怒っているとか悲しんでいる
などといった、客観的な評価が難しいコンテキストのことである。
精神的なコンテキストを集め、精神的状況をきちんと
把握することができれば多様性を持った充分な推薦ができると考えている。
そのためRive日本語入力においては精神的コンテキストをできる限り
導入していく。
現状でもキャラクターや年齢
といった精神的コンテキストは使用しているが、
いずれはカメラによるユーザーの表情解析や、
ユーザーの体温等の物理的センシングから得られる情報を
使って精神的コンテキストを推測し、
入力の多様化を担保したいと考えている。
また単語変換機能(第\ref{wordflick}項)における類語変換
や英語変換機能も
ユーザーの入力の多様性の向上に役立っている。

\section{プライバシーの問題}
Rive日本語入力においては、
入力のデータを全てコンテキストと共にサーバーに送ることで、
推薦候補を推測しているが
これはプライバシーという観点から見た時に
問題になりえるのではないかということである。

\subsection{法律的問題}
法律的にこの問題に関連しているのが「個人情報の保護に関する法律」、
通称「個人情報保護法」である。
個人情報保護法第2条1項によると
\begin{quotation}
  個人情報とは生存する個人の情報であって、
  特定の個人を識別できる情報(氏名、生年月日等)を指す。
  これには、他の情報と容易に照合することができることによって
  特定の個人を識別することができる情報(学生名簿等と照合することで
  個人を特定できるような学籍番号等)も含まれる。
\end{quotation}
とされていて、
Rive日本語入力において扱っているデータは問題がないと考えられる。
しかし入力全てをサーバーに送っているため、
「慶應義塾大学所属の臼杵壮也」という文章などを打った場合には
個人情報になってしまうという問題がある。
現状、サーバー側で個人情報にあたると思われるデータについては
学習しないようにすることで対応している。

\subsection{心理的問題}
入力の全てを送るため、
ユーザーの心理的ハードルはとても高い。
しかし入力データの送信はRive日本語入力において必要不可欠な要素であり、
現状ではユーザーに事前にデータ送信の要請をすると共に
入力データを送信しないという選択肢を与えることで対応している。

