\chapter{結論}
\label{chap:conclusion}
本章では研究において得られた結論について述べる。

\newpage
\section{研究から得られた成果}
本研究における成果を以下にまとめる。
\begin{itemize}
  \item 日本語IMEにおける様々な改善点を明らかにした。
  \item 他のIMEや外国語にも適応可能な新しいユーザーインターフェースを提唱した。
  \item コンテキストを使用するシステムの有用性を明らかにした。
  \item Rive日本語入力というシステムを設計し開発した。
\end{itemize}

\section{本システムの展望}
Rive日本語入力において情報が集まることにより、
指数関数的に使い勝手が向上すると考えている。
そのために必要なことは
アクティブユーザの確保とシステムの安定した運営のである。
まずアクティブユーザの確保を行うために、
よりユーザが使用した時の楽しさを向上させる
システムの導入を順次行っている。
また安定した運営のために、
サーバクライントモデルからより少ない通信で済むような
モデルへの変更を考えている。
またデバイスが新しいコンテキストをとることが可能になると、
本システムも自動的に性能が向上するように設計したため、
システムの有用性は向上し続けると考えている。
推薦システムにおいてはコンテキストごとにどれほど
推薦に寄与しているかを測ることで、
より精度の高い推薦ができるようになると考えている。

\section{研究への考察}
Rive日本語入力の考案から始まり、
設計、実装した上での考察である。
最新の技術を利用することで、
既存の技術や設計では成し得なかった、
システムを実装することができた。
特に推薦候補を計算し、提示するために
充分な速度を出すことができるのかは疑問であったが
充分使用に耐えうる速度での実装をすることができた。
また推薦候補単語や変換システムを使うことで、
いままでにない新しい入力へのエクスペリエンスを
ユーザに提示することができた。
ユーザインタフェースにおいては日本語にこだわらない、
多言語への適応も容易なインタフェースを提示した。
日本語IMEにはまだ多くの改善点を残し、
それに対する一つの有用なアプローチとして
研究を行うことができたと考えている。
