\chapter{結論}
\label{chap:conclusion}
本章では研究において得られた結論について述べる。

\newpage
\section{研究から得られた成果}
本研究における成果を以下にまとめる。
\begin{itemize}
  \item 日本語IMEにおける様々な改善点を明らかにした。
  \item 他のIMEや外国語にも適応可能な新しいユーザーインターフェースを提唱した。
  \item コンテキストを使用するシステムの有用性を明らかにした。
  \item Rive日本語入力というシステムを設計し開発した。
\end{itemize}

\section{システムへの評価}
Rive日本語入力は公式での配布は行っていないが
同じ研究室に所属している人や友人などに配布し
実際に使用してもらうことで、
\begin{itemize}
  \item 充分に使用可能な速度になっていること。\mbox{}\\
    IMEにおいて使用可能な速度になるように
    最新技術をふんだんに使うことで達成できた。
  \item 推薦候補が来るとローカル環境においてはすごく面白い
    と共に恐怖であるということ。\mbox{}\\
    少ない人数で運用していたため、
    身近な人々が使っている単語などが推薦され、
    推薦候補単語はとてもおもしろいものが得られたが、
    それと共に誰がどんな入力をしたのかが
    すぐにわかってしまい、
    プライベートな入力はとてもしづらい状況になってしまった。
  \item 候補単語変換機能は英語があまり得意でない人には
    とても役に立つということ。\mbox{}\\
    多様性を確保するために考案した候補単語変換機能
    であるが、入力に悩んだ時などにとても効果を発揮し、
    楽しい入力を行うことができた。
  \item コンテキストとして最も有用であるのはアクテビティー
  であるということ。\mbox{}\\
    ユーザーは入力するアプリケーションごとに、
    文体や語句を使い分けることが多く、
    アクテビティーによるコンテキストを用いた
    推薦が最も効果を発揮しているということがわかった。
\end{itemize}
といった結果を得ることができた。

\section{本システムの展望}
Rive日本語入力において情報が集まることにより、
指数関数的に使い勝手が向上すると考えている。
そのために必要なことは
アクティブユーザの確保とシステムの安定した運営である。
まずアクティブユーザの確保を行うために、
よりユーザが使用した時の楽しさを向上させる
システムの導入を順次行っている。
また安定した運営のために、
サーバクライントモデルからより少ない通信で済むような
モデルへの変更を検討している。
またデバイスが新しいコンテキストの取得が可能になれば、
推薦の精度が多少なりとも向上するため、
システムの有用性はデバイスの進化と共に上がると考えている。
推薦システムにおいてはコンテキストごとにどれほど
推薦に寄与しているかを測ることで、
より精度の高い推薦ができるようになると考えている。

\section{研究への考察}
ここではRive日本語入力の考案から始まり、
設計・実装した上での考察を行う。
最新の技術を利用することで、
旧来の技術や設計では成し得なかった、
システムを実装することができた。
特に推薦候補を計算し、提示するために
充分な速度を出すことができるのかは疑問であったが
充分使用に耐えうる速度での実装をすることができた。
また推薦候補単語や変換システムを使うことで、
いままでにない新しい入力へのエクスペリエンスを
ユーザに提示することができた。
ユーザインタフェースにおいては日本語にこだわらない、
多言語への適応も容易なインタフェースを提示した。
日本語IMEにはまだ多くの改善点を残し、
それに対する一つの有用なアプローチとして
研究を行うことができたと考えている。
